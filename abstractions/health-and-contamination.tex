\subsection{Healing}\label{subsec:healing}
For living entities damage is represented by wounds with the severity being the remainder of the damage after all
defenses were subtracted.
Humans roll a Fitness-Check once a wound starts healing.
For every met threshhold of the wound one level of regeneration rate for that particular wound is noted down.
If a wound is worsened/increased the Fitness-Check is repeated, selecting the worse of the two results.
If a wound is successfully and substantially improved, the Fitness Check is repeated, selecting the better of the two
results.
Regeneration can be aided or slowed by circumstances, such as therapy or environmental conditions.
Resonance of frequency 1 lowers the healing rate by its amplitude, even going negative, while resonance of frequency 10
improves healing rate by its amplitude.
Wounds do not necessarily lead to death or permanent impairments, but if they remain untreated for a long time, or there
is a runaway effect on one of them, the Storyteller may decree permanent repercussions, as appropriate, including Death.
Regeneration accumulates over successive days, and once the current severity is reached, the wound is lowered by 1
severity and regeneration is reset to 0.
If regeneration rate is negative, and the regeneration rate falls below 0, regeneration progress is set to the wound
severity and then the severity is increased by 1.
Standard threshholds are 2, 4, 6, 8, 10, 12, 14. \par
The following table lists healing time in days for severity and hit threshholds.\par
\begin{tabular}{c|rrrrrrrrrr}
    Severity & 1 & 2 & 3 & 4 & 5 & 6 & 7 & 8 & 9 & 10\\
    1 &  1 d & 1 d & 1 d & 1 d & 1 d & 1 d & 1 d & 1 d & 1 d & 1 d\\
    2 &  3 d & 2 d & 2 d & 2 d & 2 d & 2 d & 2 d & 2 d & 2 d & 2 d\\
    3 &  6 d & 4 d & 3 d & 3 d & 3 d & 3 d & 3 d & 3 d & 3 d & 3 d\\
    4 &  10 d & 6 d & 5 d & 4 d & 4 d & 4 d & 4 d & 4 d & 4 d & 4 d\\
    5 &  15 d & 9 d & 7 d & 6 d & 5 d & 5 d & 5 d & 5 d & 5 d & 5 d\\
    6 &  21 d & 12 d & 9 d & 8 d & 7 d & 6 d & 6 d & 6 d & 6 d & 6 d\\
    7 &  28 d & 16 d & 12 d & 10 d & 9 d & 8 d & 7 d & 7 d & 7 d & 7 d\\
    8 &  36 d & 20 d & 15 d & 12 d & 11 d & 10 d & 9 d & 8 d & 8 d & 8 d\\
    9 &  45 d & 25 d & 18 d & 15 d & 13 d & 12 d & 11 d & 10 d & 9 d & 9 d\\
    10 &  55 d & 30 d & 22 d & 18 d & 15 d & 14 d & 13 d & 12 d & 11 d & 10 d\\
    11 &  66 d & 36 d & 26 d & 21 d & 18 d & 16 d & 15 d & 14 d & 13 d & 12 d\\
    12 &  78 d & 42 d & 30 d & 24 d & 21 d & 18 d & 17 d & 16 d & 15 d & 14 d\\
    13 &  91 d & 49 d & 35 d & 28 d & 24 d & 21 d & 19 d & 18 d & 17 d & 16 d\\
    14 &  105 d & 56 d & 40 d & 32 d & 27 d & 24 d & 21 d & 20 d & 19 d & 18 d\\
    15 &  120 d & 64 d & 45 d & 36 d & 30 d & 27 d & 24 d & 22 d & 21 d & 20 d\\
    16 &  136 d & 72 d & 51 d & 40 d & 34 d & 30 d & 27 d & 24 d & 23 d & 22 d\\
    17 &  153 d & 81 d & 57 d & 45 d & 38 d & 33 d & 30 d & 27 d & 25 d & 24 d\\
    18 &  171 d & 90 d & 63 d & 50 d & 42 d & 36 d & 33 d & 30 d & 27 d & 26 d\\
    19 &  190 d & 100 d & 70 d & 55 d & 46 d & 40 d & 36 d & 33 d & 30 d & 28 d\\
    20 &  210 d & 110 d & 77 d & 60 d & 50 d & 44 d & 39 d & 36 d & 33 d & 30 d\\
    21 &  231 d & 121 d & 84 d & 66 d & 55 d & 48 d & 42 d & 39 d & 36 d & 33 d\\
    22 &  253 d & 132 d & 92 d & 72 d & 60 d & 52 d & 46 d & 42 d & 39 d & 36 d\\
    23 &  276 d & 144 d & 100 d & 78 d & 65 d & 56 d & 50 d & 45 d & 42 d & 39 d\\
    24 &  300 d & 156 d & 108 d & 84 d & 70 d & 60 d & 54 d & 48 d & 45 d & 42 d\\
    25 &  325 d & 169 d & 117 d & 91 d & 75 d & 65 d & 58 d & 52 d & 48 d & 45 d\\
\end{tabular}
\subsection{Treatment}\label{subsec:treatment}
Treating of wounds is usually done with a Competence or Theory + Healing Check, where medicine and equipment
can provide modifiers or advantages.
Many wounds require medicine to be treated at all.
Treating them without will either not be possible or generate disadvantage.
Standard threshholds are 8, 11, 14, 17, 20, with results below 5 having negative consequences.

Treating ingress wounds requires at least 10g of Alacast, with the treatment of
contamination inflicting the sum of all dice including and below or equal to a threshhold determined by the
Technology used (i.e.\ 8 for Saline-Alacast-Solution Injection) as damage.
The level of ingress wounds and contamination is directly lowered by the number of hit threshholds.

In all cases Alacast is used to precipitate Contamination from the body, the
crudest way is to introduce Alacast to the bloodstream to precipitate it inside the body and hope the body expels
it on its own, more refined ways are basically a dialysis or a very specific targetting of crystallization seeds
and retrieval of precipitation clusters.

\subsection{Medicine}\label{subsec:medicine}
Medicine is used while making a Treatment Check.
Each Treatment of a wound using Medicine consumes
severity\(\cdot\)100g of the medicine and provides the specific bonus.

\begin{enumerate}[label= -]
    \item \textbf{Naturopathy} uses healing and the respective field of the medicine.
    It provides a good floor and good threshhold bonuses.
    \item \textbf{Specialized Medicine}(name pending) live mixing of specialized cures from ingredients
    uses healing and the respective field of the ailment.
    It has no floor, but scales the best.
    \item \textbf{Broadband Medicine} is mass produced, refined medicine and uses a com\-bi\-nation of
    Healing, \-Scie\-nce(Med\-icine) and \-Red Biotech.
    It has a good \-floor, but low \-scaling.
    \item \textbf{Alacast} has no direct medicinal benefit, but effectively combats Contamination in all fields
    \item \textbf{Elixir of Life} is the top notch medicine, rejuvinates and heals nearly anything.
    Sadly, it contaminates the user.
\end{enumerate}

\subsection{Categories}\label{subsec:categories}
Every Character has a Contamination Resistance rating ranging from 1 to 5,
but in-world they are categorized (by supervivo) from A to C\@.
\begin{enumerate}[label = - ]
\item \textbf{CatA} Humans are usually Supervivo and usually have low Contamination Resistance
                    and high Affinity.
\item \textbf{CatB}s are outlaws, traders, hermits or for some other reason living outside the Domes and Tribes.
                    They can have any, but usually have medium Contamination Resistance and medium Affinity
\item \textbf{CatCs} are usually Tribals with high Contaminatin Resistance and low Affinity.
\end{enumerate}
\begin{tabular}{c|cc}
    Entity & Contamination Resistance  & Affinity requirements\\
    Experimental & 1 & 5 \\
    High Tech & 2 & 4\\
    Mid Tech & 3 & 3 \\
    Low Tech & 4 & 2\\
    Base Tech & 5 & 1\\
\end{tabular}

\textbf{Affinity requirements} are the base amount of Affinity that a character has to have to use the Entity.
If the requirements are unmet, every use incurs the difference in penalty dice.


\subsection{Contamination}\label{subsec:contamination}
Contamination is a term used to describe the amalgamation of \-technology destroying \-nanoweapons, \-radiation, \-pollution
and general \-environmental \-hazards.
Contamination is air\-borne, waterborne, bloodborne and per\-meates pretty much everything in the environment.
It is usually assigned a \-level \-describing its intensity.
Only within the Domes of the Supervivo, \-inside a Mech or in a similarily \-purified environment can the \-
Contamination\-level ever reach 0. \par
Levels below 0 are necessary for taking apart Experimental Tech Technology, but getting an area so pure is hard.\par
\begin{enumerate}[label = - ]
\item Levels 1--3
are rare areas of low Contamination, like mines, airlocks, outdoor markets, outdoor settlements and so on.
Even subjected to this level of Contamination for months, it will not cause death.
However, for the more fragile Members of Society, the general quality of life can be greatly diminished.
\item Levels 4--6
are the most common and usually inhabited by Cat-B\@.
People without a form of permanent resistance can die here, but life expectancies should be a few months at least.
\item Levels 7--11
are usually inhabited by tribals but sometimes a few Cat-B have to make Camp here.
Anyone without contamination protection will die here within a few weeks.
\item Levels 12--15
are sparsely populated by hardy Tribals, but usually deserted.
\item Levels 16+
are the most hostile areas.
Weird Things happen here, but some say that there is great treasure\ldots
\end{enumerate}
There is no upper limit for contamination levels.
If an entity is in an area with contamination, every hour it receives the local contamination
as contamination damage.\par

\subsection{Contamination damage}\label{subsec:contamination-damage}
Anytime the an entity receives contamination damage, the character lowers that damage with a roll on their contamination
resistance and fitness plus all applicable modifiers (external only for contamination ingress).
Remaining Damage above 0 is applied to the ingress wound (extending the existing ingress wound if possible,
otherwise creating a new one) and raises contamination of that character by 1.
The contamination of the character is applied as an internal bonus on the contamination resistance result, but reduces
their healing rate, including going negative.



\subsubsection{Contamination Damage to Entities}
When a character directly interacts with an entity, the Charactercontamination\(\times\)2
of that character is applied as Contamination damage.
This is repeated every hour if the interaction continues.
While directly contacting an entity only the internal resistance applies, direct contact might be piloting a mech,
using a gun or shaking someones hand, except when proper seals are in place.

Instead of Fitness an object uses how well adapted it is to being in contact with contamination.
This is a combination of relevant factors.
In general the more artificial an Object is the easier it is broken down,
the more regular it is the easier it is disturbed and the more complicated it is the easier it is to bring into dissaray.

Some Examples would be

\begin{enumerate}
    \item Rock/Stick/Clump of Dirt: might bear corruption but is usually not affected
    \item Bronze Sword (5,5): the Edge will dull in days or weeks and the material will acquire a thick Patina,
    but over all it is quite usable and with the patina sealing it, it would last a long time.
    \item Combustion Motor(4,4): a few hours in Corrupted air and it will deteriorate
    \item Vaccuum Tubes(3,4): from the perspective of Corruption ``just'' glass, but all the interesting parts around it
    \item High Tech Supervivo Survival suit(2,5): Made from simple, chaotic polymers, it can
    withstand Corruption even without a Sealant for several days before crumbling
    \item 3nm Hightech Processor(0,2): Vulnerable on every front.
\end{enumerate}


Damaged Objects might loose some or all of their functionality, or cause penalty dice when using them.

\subsection{Seals}\label{subsec:seals}
A seal of a level of X lowers the level of the Contamination behind it by X
If a seal is damaged, its effectiveness is lowered by 1 and until it is patched, it only counts for half
its current value

Air has to be purified separately or enriched with oxygen.
(One human consumes about 500--600 litres of oxygen per day.)
The Domes of the Supervivo for
example usually have at least a level 25 hermetical seal with purified and oxygen enriched air.

Examples of human sized external contamination modifiers or in-place options:

\begin{tabular}{cl}
    1& An alacast infused rag or a shut wooden door (although oxygen may run out)\\
    2& a breathermask or a simple wooden door airlock (again, oxygen)\\
    3& a partial gasmask or an airtight airlock\\
    4& a heavy partial gasmask with Goggles or a basetech air filtering facility\\
    5& a full gasmask\\
    6& as above + sealed clothing or lowtech air filtering facility\\
    7& as above + midtech scrubber (a little backpack)\\
    8& as above + hightech air scrubber or midtech air filtering facility\\
    9& as above + full mask with air from a compressed air tank\\
    10+& \hyperref[subsec:seals]{specially sealed} suit with airtank or hightech filtering facility\\
\end{tabular}
