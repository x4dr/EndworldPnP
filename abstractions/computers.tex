\subsection{Computers}\label{subsec:computers}
Computers come in many different shapes and sizes, but in Endworld they are all abstracted into
their availaible memory and their available processing powers.

Computers in Endworld are different from our own, since they had to be designed to be much easier, much more
modular and much more resistant, they are usually quite big and heavy, but also incredibly sophisticatedly
parallelized.
Also they are heavily abstracted and should not be confused with real computers.

\subsubsection{Processing Power}
Processing power is a computers capability to achieve computation and the general work, including I/O\@, it is measured
in Computational Power Usage, and a Processor provied CPU capacity.
Processing power is occupied by programs, taking as much as they need or failing to start if there is not enough.
A Theory, Computer Usage (and relevant skill for computer administration, clocking or similar) Roll can be used to shift 
[8,12,17,20] processing from one program to another or free that processing power up.
A program running below its required processing power, will perform worse.
Scale any applicable metric to the utilized processing power, rounding unfavorably for purposes of the program.
Programs with no direct metric, still produce lesser effects, or may not work at all.
A program running above its required processing power usually does not do its tasks better, but might perform them
faster.

\subsubsection{Storage Space}
Storage space is a computers capability to store data and programs.
It is measured in Storage.

\subsubsection{Programs}
For any Program to run, it needs to either be loaded from storage or be present as built in RROM\@.

\paragraph{RROM}
Robust Read Only Memory is a storage medium directly integrated into a processor and by custom usually made from non
volatile, corruption resistant materials like woven strands of wire and beads.
Since a program is needed to load a program, some processors have built in RROM acting as a bootstrap loader for the
Main program used to load other programs more efficiently.
Since the RROM is Using its own Computational Power it can not be reused and the Bootloader is effectively its own
Computersystem.
It starts executing as soon as it is turned on and the program can not be changed or modified while the Processor is
running.
However simply physically replacing the program is enough and so RROM has a big advantage over hardware or even elektro-
mechanic computing techniques.

\paragraph{Execution}
Programs have a set of goals that they will try to accomplish and through their programming take steps towards.
This culminates in the abstracted behaviour of programs performing actions, some with a predetermined degree of success,
some provide parts or all of the statistics needed for a check.
Programs can load other programs either in parallel (while they are still running) or in series
(as they are completing).
When a program is completed, it unloads and at the end of the turn any ressources it used are freed.

\paragraph{Operating System}
An Operating system or ``Main'' is simply another program that can be dynamically tasked by the user to load other
programs.
A user interaction requires usually one turn, after which the Operating system performs a predetermined series of loads
and supplies each program with its parameters.

\paragraph{Parameters}
Are the context a specific program needs to accomplish its goal.
Parameters are usually abstracted away as long as it is reasonable to assume that all programs ``know'' what they
are acting upon.

\subsubsection{Programming}
To create a Program requires some knowledge of what the Program is trying to accomplish, and Computer Programming and
usually Theory.
If the process is more exploratory and there is no clear goal, Insight is used instead to merely
determine if a solution can be found.
For Theory checks, given that the programmer could conceivably find a series of Instructions, The Storyteller and
Player determine a set of CPU requirements, Storage space needed for the Program, auxiliary data requirements, flat
Checkresults or purposeful statistics, other programs loaded and runtime, as far as applicable for that program, with
similar Programs and their checked quality as templates.
After the Check was made, the Storyteller offers several configurations of the above values that could be achieved with
the roll and the Player chooses one.
Alternatively the Player designs a new Program, its configuration and effects and the Storyteller tells the Player the
interval of the programming rolls and the total sum of checkresults needed to achieve the desired goal.
The program, its name, its statistics and effects are then noted down by the Player.

\paragraph{Example - Shield Daemon}
\begin{enumerate}
    \item Name: Shield Daemon
    \item CPU: 0.5
    \item Size: 1
    \item Requirement: Internal Computerization, Shields
    \item Effect: If a switch inside the cockpit is flipped to ``bring shields up'',
    load one Shield Configurator for each shield that is down,
    if it is flipped to ``NOW'', load one Shield Burster for each shield that is down
    \item Runtime: Ongoing
    \item To program: Shields or relevant piloting skill, 1 workday per roll, total: 100
    \item Limits: specific to exact programs and mech
\end{enumerate}

\paragraph{Example - Shield Configurator}
\begin{enumerate}
    \item Name: Shield Configurator
    \item CPU: 3
    \item Size: 1
    \item Requirement: Shield
    \item Effect: Configure Shield with Flat Result of 10
    \item Runtime: 1 Turn
    \item To program: Shields, 1 workday per roll, Any 3 Rolls of 16+
    \item Limits: specific to exact shield
\end{enumerate}

\paragraph{Example - Shield Burst}
\begin{enumerate}
    \item Name: Shield Burster
    \item CPU: 1
    \item Size: 1
    \item Requirement: Shield
    \item Effect: Configure Shield with Flat Result of 0
    \item Runtime: Instantly
    \item To program: Shields, 1 workday per roll, Total of 30+
    \item Limits: specific to exact shield
\end{enumerate}
