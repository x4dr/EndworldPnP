\section{Life}\label{sec:life}
If an entity gets damaged, the damage is decreased by the result of the roll made by the outermost layer of defense.
If the remaining damage is above 0, that remaining damage is usually passed on to the layer below.
Layers can fail, and then no longer participate during damage prevention.
If damage passes all layers of defense,
the remaining numerical value of the damage is noted down as the severity of the damage.
For a Biological Entity damage is wounds an may heal..
\section{Death}\label{sec:death}
There is no fixed numerical amount of lifepoints, but every wound/instance of damage will impede the Entity.
This can manifest in many ways, like penalty dice for associated activities, higher botch/malfunction frequencies,
and many more.
If there is a lot of damage, machines may become inoperable, people might pass out.
Death and Destruction only occur when the Gamemaster decrees so.
\section{Rounds}\label{sec:rounds}
Combat is done in rounds.
Each round is 5 seconds long.
Every character present acts in turn (determined by circumstances or a Stealth, Perception, Willpower or other
situational check) and can usually move and do an action (an attack or action that can be done in the remaining time).
\section{Detailed combat}\label{sec:detailed-combat}
\begin{enumerate}
    \item Every participant (technically in secret) decides on what they are going to do.
    \item TODO: make words good: Every participant rolls the worst applicable roll of any actions they plan to undertake in that round at once.
    \item Each defense is determined, depending on the current action of the participant.
    \item Each offense is determined, depending on the currect action and target of the participant.
    \item Non-offensive Actions are resolved first, then offensive actions in descending Order of rolled result.
\end{enumerate}

\subsection{Defense}\label{subsec:defense}
Defense is usually checked with Agility and
\begin{enumerate}[label = - ]
    \item \hyperref[subsec:tactics]{\textbf{tactics}} on ordered advance/retreat
    \item \hyperref[subsec:instinct]{\textbf{instinct}} in general mayhem
    \item \hyperref[subsec:footwork]{\textbf{footwork}} in close range
    \item \hyperref[subsec:running]{\textbf{running}} when zig zagging
\end{enumerate}
Evasion levels are determined with the result of the defensive interpretation of the combat roll
and the character specific evasive threshhold set.

\subsection{Offense}\label{subsec:offense}
Offense level are determined with the result of the offensive \-interpretation of the combat roll \-and the range and
weapon \-specific threshhold set.\par
Aiming Penaltydice
\begin{tabular}{cc}
    target & penalty\\\hline
    head & 3 \\
    hand & 2 \\
    legs & 1 \\
\end{tabular}
\subsection{Hit Calculation}\label{subsec:hit}
If the offense level of an attack is at least 1, the hit roll is made.
It consists of a singular die, with the offense levels as bonus dice, and the defense levels as penalty dice,
see \hyperref[sec:bonus--and-penaltydice]{Bonus- and Penaltydice}.
The hit quality will be a number from 1 to 10. \par
\begin{tabular}{c|cr}
    hit quality & result & damage modifier \\\hline
    1 & far miss & 0\%\\
    2 & miss& 0\%\\
    3 & close miss& 0\%\\
    4 & near miss& 0\%\\
    5 & tangential& 1\%\\
    6 & graze& 10\%\\
    7 & slight hit& 20\%\\
    8 & partial hit& 50\%\\
    9 & hit& 100\%\\
    10 & full hit& 100\%\\
\end{tabular}

On a full hit, the attacker inflicts the wound were planned
The specific damage to apply will be listed with the weapon.
