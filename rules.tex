\section{The Selector System}\label{sec:selectorsystem}
This System uses  ten sided dice.
A Check is a roll of usually 5 dice interpreted in a certain way,
with the result describing how successfull the attempted action was.\par
Whenever there is advantage or disadvantage of any kind,
\hyperref[sec:bonus--and-penaltydice]{Bonus- or Penaltydice} may be applicable.
The Roll itself is interpreted by your \emph{Selectors} which usually are defined
by the applicable Attribute and Skill, but may be static, or Equipment based.
The Dice in the Roll are ordered and given an order-number in ascending order by value.
To get the Result of the Roll (or to \emph{interpret} the roll), every selector \emph{selects} a die with the
appropriate order-number and adds them to the result.
Selectors greater than the biggest order-number simply select the highest die.
A Selector of 0 selects no die.
The usual number of Selectors is 2.\par
The effects and result of the Check may then be further modified by \hyperref[sec:resonance]{Resonance}.
\vspace{1cm}

\section{Bonus- and Penaltydice}\label{sec:bonus--and-penaltydice}
Bonus and Penaltydice cancel each other out, so 3 Bonusdice and 2 Penaltydice results in 1 Bonusdie.\par
They describe how many extra dice are rolled on a given Roll.
If Bonusdice were rolled,
remove the lowest dice from the Roll until the number of dice is as it was before the extra dice.
For Penaltydice, the highest are removed instead.
Neither Bonus nor Penaltydice ever change the number of dice being interpreted, just the relative Chances.

\section{Resonance}\label{sec:resonance}
Resonance is when in a Roll, more than one die, show the same Number.\par
The Resonance \emph{amplitude} is the number of dice \emph{that exceed 1}, meaning it is lower by 1 than the
number of dice.\par
The Resonance \emph{frequency} is the number those dice show.
If no Effect states anything about a Resonance, they have no effect.

\chapter{Skills and Attributes}\label{ch:skillsandattributes}
\section{Physical Skills}\label{sec:strength-skills}
\subsection{Heavy Athletics}\label{subsec:heavy-athletics}
do you even lift?
\subsection{Running}\label{subsec:running}
running fast is usually fitness, but agility may come in if it is running to avoid being shot at
\subsection{Jumping}\label{subsec:jumping}
leaping high or far, this one has to do with energy bursts from the legs
\subsection{Throwing}\label{subsec:throwing}
yeet
\subsection{Survival}\label{subsec:survival}
wilderness skills
\subsection{Acrobatics}\label{subsec:acrobatics}
Uses include: doging from cover to cover, backflips, climbing, balancing and many others
\subsection{Weapon()}\label{subsec:weapon}
There are many weapons to choose from!
\subsection{Footwork}\label{subsec:footwork}
Used mainly in close range, \textbf{footwork} describes the characters ability to move fast and decisively in combat,
outmaneuvering would be enemies

\section{Social Skills}\label{sec:social-skills}
\subsection{Trade}\label{subsec:trade}
trading is used to \hyperref[ch:trade]{Trade}
\subsection{Diplomacy}\label{subsec:diplomacy}
finding solutions that both sides want
\subsection{Misdirection}\label{subsec:misdirection}
includes lying and general providing of wrong information as correct
\subsection{Style}\label{subsec:style}
panache!
style!
generally being impressive/following characters style
\subsection{Intimidation}\label{subsec:intimidation}
\subsection{Rhethoric}\label{subsec:rhethoric}
Words, I have the best Words
\subsection{Command}\label{subsec:command}
giving orders to others
\subsection{Empathy}\label{subsec:empathy}
knowing how others feel
\subsection{Etiquette}\label{subsec:etiquette}
Knowing how to behave
\subsection{Rumor}\label{subsec:rumor}
the skill at navigating the rumor mill, sifting through verbally repeated information and even starting rumors
\subsection{Art()}\label{subsec:art}
may be any art that is used primarily for self expression and social interaction.

\section{Mental Skills}\label{sec:wisdom-skills}
\subsection{Healing}\label{subsec:healing-skill}
important skill for \hyperref[subsec:treatment]{Treating of Wounds}
\subsection{Research}\label{subsec:research}
acquiring and processing information from readily available sources
\subsection{Instinct}\label{subsec:instinct}
used for sensing things that are not being paid attention to and as situational awareness in combat
\subsection{Search}\label{subsec:search}
user for sensing things specifically declared by the player or looked for
\subsection{Strategy}\label{subsec:strategy}
larger ``zoomed out'' decisions and their repercussions
\subsection{Tactics}\label{subsec:tactics}
local ``zoomed in'' decisions and their repercussions
also sometimes used with agility for advancing / retreating in a firefight
\subsection{Navigation}\label{subsec:navigation}
if you need to know where you are and how to get to a place
\subsection{Knowledge()}\label{subsec:knowledge}
lore or science, any sort of mostly theoretical knowledge

\section{Ability Skills}\label{sec:ability-skills}
\subsection{Biotech}\label{subsec:biotech}
Biotech is the Fusion of Biology and Technology \par
The ancient texts define Biotechnology as "any technological application that uses biological systems, living organisms,
or derivatives thereof, to make or modify products or processes for specific use".
Biotechnology is such a diverse group of skills, that each field is ungrouped rules-wise.
\begin{enumerate}[label= -]
    \item \textbf{Red} Biotechnology Encompasses genetics, genomics, pharmacology, but also medical advances like
    genetherapy and automedics and deconamination of living things.
    \item \textbf{Green} Biotechnology is used every time the passive environment is exploited,
    from engineering high yield, low footprint farming to technologically harnessing the great and small Anomalies,
    including detoxification of materials and beings.
    \item \textbf{Blue} Biotechnology has developed from being used for aquatic Biotech, to include Megafauna, since the
    first Megafauna is told to having been spottet in the ocean.
    It allows scanning, categorization and assessment of Megafauna, including in some cases,
    harvesting Megafauna specialities.
    \item \textbf{White} or Industrial Biotechnology  is used in Industrial processes and in handling some of the
    results of such processes.
    Creating and applying contamination-sealant is one of such processes.
    \item \textbf{Black} deals with with integration of Biology and Technology from enhancements or replacements like
    Cyberware or more conventional Prosthetics to synthesizing life, ressurection and strong AI. Seldomly useful
    and the most dangerous brand of Biotechnology, it is not taught often.
\end{enumerate}
\subsection{Computer}\label{subsec:computer}
The Computer skill group describes the ability and training of the character to generally interact with computers.
\begin{enumerate}[label= -]
    \item \textbf{Programming} is creating or changing computer programs.
    \item \textbf{Usage} is using complex features of computers in intended ways.
    \item \textbf{Hacking} is making computers behave in unintended ways.
\end{enumerate}
\subsection{Engineering}\label{subsec:engineering}
The Engineering skill group describes the ability and training of the 
character to generally create and care for machines.
\begin{enumerate}[label= -]
    \item \textbf{Repair} 
    \item \textbf{Design} 
    \item \textbf{Build} 
\end{enumerate}
\subsection{Aux}\label{subsec:aux}
Auxiliary Training is training with specific types of equipment and tactics.
This includes operation and maintenance of these devices.
\subsubsection{Defense}
Skillgroup pertaining to defensive Equipment options.
\begin{enumerate}[label= -]
    \item \textbf{Decoys} From Flares against heat seeking missiles to Holographic Projectors, anything that guides
     an opponents Weapon to where it can inflict no harm.
    \item \textbf{Shields} Devices that dampen or stop kinetic energies, especially those of weapons
    \item \textbf{Deflectors} Anything that redirects energies, especially those of weapons
\end{enumerate}
\subsubsection{Offense}
Skillgroup pertaining offensive Equipment and Tactics
\begin{enumerate}[label= -]
    \item \textbf{Explosives} Training with things like mines, piles of C4 and IEDs
    \item \textbf{Sniping} Equipment and Tactics to specifically target enemy Systems and increase accuracy
    over rate of firing
    \item \textbf{Gadgets} Hidden weapons, fireable homing beacons, Flare guns, Flamethrowers, Cryothrowers, Gas
\end{enumerate}
\subsubsection{Recon}
Skillgroup pertaining to gathering information while remaining unseen
\begin{enumerate}[label= -]
    \item \textbf{Scanners} Equipment to obtain Information, from Sensors over Radar to Compositionscanners
    \item \textbf{Suppressors} Stealth equipment from mecha-weapon compatible flash or noise suppressors to thermal
    signature dissipating gadgets and even invisibility ``cloaks''
    \item \textbf{Sabotage} EMP, ECM, ECCM, but also poison, pitfalls and similar tactics
\end{enumerate}
\subsubsection{Utility}
Skillgroup pertaining to opening up more options
\begin{enumerate}[label= -]
    \item \textbf{Communications} Equipment with the purpose of transferring Information
    \item \textbf{Reprocessing} Equipment to turn things into other, generally more useful things.
    From electrolysis to cooking.
    \item \textbf{Utility Movement} Equipment that allows movement in nonstandard ways, such as flying, climbing,
    swimming, boring or even teleporting
\end{enumerate}
\subsection{Professions/Proficiencies}\label{subsec:professions}
each of these are a Skillgroup, meaning after the initial point, further progress needs to be specialized.
The exact nature of these specializations is open to player and storyteller.
\subsubsection{Metal}\label{subsubsec:metal}
\subsubsection{Ceramics}\label{subsubsec:ceramics}
\subsubsection{Leatherworking}\label{subsubsec:leatherworking}
\subsubsection{Boneworking}\label{subsubsec:boneworking}
\subsubsection{Woodworking}\label{subsubsec:woodworking}
\subsubsection{Woodcraft}\label{subsubsec:woodcraft}
\subsubsection{Animalcare}\label{subsubsec:animalcare}

//TODO: General format and make words good yes.
Every sublayer has up to 3 levels, only one sublayer can ever be on level 3, the rest are capped at level 2,
transfer can be done.
Every superlayer has up to 3 levels, and can only be increasedby 1 for every related sublayer that is at level 2 or 3
\subsection{Mecha Weapons Expertise}\label{subsec:mecha-weapons-expertise}
Mecha Weapons means any weapon too big to be carried, including stationary turrets.
Usually Weapons are fired with \hyperref[itm:focus]{Focus}.
\subsubsection{Laser}
Laser or technically "Directed Energy Weapons" \par
are a group of weapons which typically do not use Ammunition, but raw Energy to inflict their damage.
Their comparably low damage is made up for by being as accurate as a light beam,
plus, the high amounts of particles in the air make for pretty lights.
\subsubsection{Projectile}
Projectile Weapons or sometimes just "Guns" \par
are a group of weapons which discharge Projectiles towards a target.
Ammunition has to be employed, but there are several different types available.
\subsubsection{Missile}
Missile Weapons are a weapons that travel to the target to inflict damage, the difference
to Projectile Weapons is that Missile Weapons are usually self propelled instead of being accelerated at their
startpoint.
Missiles are comparatively heavy and expensive, but most of them ignore shields and/or deliver
special devices or bombs to the target.
\subsubsection{Attack Modes}
\begin{enumerate}[label= -]
    \item \textbf{Direct} Weapons are many and varied but share the characteristic of being pointed roughly at
    the target they are shooting.
    They require line of sight.
    \item \textbf{Indirect} Weapons are just as varied as Direct Weapons but have the key difference
    of not pointing at the target.
    Usually Up and in an angle, but not always.
    They require information about the targets position, but are free from line of sight restrictions, as long as
    the Attackstill has a valid way to travel to the target.
    \item \textbf{Seeking} Weapons have some sort of sensors or a dataconnection, and will follow a Target, as long as the Lock
    remains unbroken.
    This skill mainly deals with the correct usage and parametrization of such Weapons.
    \item \textbf{Beam} Weapons that emit long, sometimes even sustained streams, making them easier to aim
    and inflicting more damage, usually at the price of higher, sustained costs and less flexibility.
\end{enumerate}
\subsubsection{Melee}
Melee Weapons of mech sizes are capable of inflicting utter destruction, but are usually
limited to close ranges, which makes them not very useful at long distances.
\begin{enumerate}[label= -]
    \item \textbf{Sharp} Melee Weapons have slightly longer range and are faster and usually do good damage.
    Most of them use the Movement System to get ready to attack again, so they may attack a lot of times on fast mechs.
    They also are the most precise of the melee weapons.
    \item \textbf{Blunt} Melee Weapons are usually slow, and easier to dodge, but their raw damage output is only
    matched by Artillery.
    \item \textbf{Unarmed} is not technically a category of weapon, but it represents being able to fight in melee
    without any special weapon, which allows for mostly disabling maneuvers.
\end{enumerate}
\subsection{Piloting Practice}\label{subsec:piloting-practice}
\begin{enumerate}[label= -]
    \item \textbf{Plain} Terrain
    \item \textbf{Desert} Terrain
    \item \textbf{Forest} Terrain
    \item \textbf{Arctic} Terrain
    \item \textbf{Mountain} Terrain
    \item \textbf{Urban} Terrain
    \item \textbf{Mud} Terrain
    \item \textbf{Marine} Terrain
    \item \textbf{Extreme} Terrain
    \item \textbf{Exo} is short for "Exoskeleton" and refers to Mechs that are barely larger than an extended Bodyarmor
    or Mechs that are generally not much bigger than their Pilot. (<1t)
    \item \textbf{Tiny} a versatile spot for Contraptions between Exo and Very Light, these include buggies, cars and
    a lot of small, fast Vehicles. (1-5t)
    \item \textbf{Very Light} include things like APCs and small fast walkers and light tanks.(5-15t)
    \item \textbf{Light} is about as big as a medium to main tank. (15-50 t)
    \item \textbf{Medium} is about as big as a heavy to superheavy tanks (50-150 t)
    \item \textbf{Heavy} is above superheavy tanks. (150-500t)
    \item \textbf{Very Heavy} LANDKREUZER (500t-1500t)
    \item \textbf{Ultra} ???
\end{enumerate}
\subsection{Aux}\label{subsec:aux}
\subsubsection{Defense}
\begin{enumerate}[label= -]
    \item \textbf{Decoys}
    \item \textbf{Shields}
    \item \textbf{Deflectors}
\end{enumerate}
\subsubsection{Offense}
\begin{enumerate}[label= -]
    \item \textbf{Explosives}
    \item \textbf{Sniping}
    \item \textbf{Gadgets}
\end{enumerate}
\subsubsection{Recon}
\begin{enumerate}[label= -]
    \item \textbf{Scanners}
    \item \textbf{Suppressors}
    \item \textbf{Sabotage}
\end{enumerate}
\subsubsection{Utility}
\begin{enumerate}[label= -]
    \item \textbf{Communications}
    \item \textbf{Reprocessing}
    \item \textbf{Movement}
\end{enumerate}
\par
\section{Spiritual Skills, Mageside}\label{sec:spiritual-skills,-magicside}
Magic is to be determined with the magicuser.

\subsection{Raven}\label{subsec:raven}
This skillgroup is an example of a Tribal Raven shamans link with their totem
\begin{enumerate}[label= - ]
    \item \textbf{Talon} generally damaging and interacting spells
    \item \textbf{Eye} scouting and information spells
    \item \textbf{Wing} movement and utility spells
\end{enumerate}

\subsection{Spirit}\label{subsec:spirit}
This skillgroup is an example of a Tribal Raven shamans general ability to manipulate Contamination
\begin{enumerate}[label= - ]
    \item \textbf{Call} increases Contamination at target (further away => weaker effect)
    \item \textbf{Infuse} increases Contamination in target and restores it
    \item \textbf{Expunge} decreases Contamination in target and damages it
\end{enumerate}


\subsection{Resonant Body}\label{subsec:resonant-body}
This skillgroup is an example of a Tribal Warriors awoken innate superhuman abilities.
\begin{enumerate}[label= - ]
    \item \textbf{Resonant Healing} Every 10-Discipline (maximum Resonant Healing) Contamination
    provides a bonus die when healing.
    \item \textbf{Tireless} Focus negates exhaustion, requires Contamination lasts 1/4/16 hours per day
    \item \textbf{Ancestors Strength} Intuition Adds to Physical skill in times of need (Ancestors Strength times per day)
\end{enumerate}

\subsection{Deathless}\label{subsec:deathless}
This perkline provides the legendary resilience seen in some Tribals
\begin{enumerate}
    \item \textbf{Last Chance} No Single Wound can knock the character out
    \item \textbf{Return} If the Body is restored soon after death, it will come back to life.
    \item \textbf{Join the Ancestors} On Death Character becomes a ghost and is able to communicate with receptive minds.
\end{enumerate}


\subsection{Generalized, Vague, Pointsink}\label{subsec:GVPS}





\chapter{Trade}\label{ch:trade}
These entries in the table below are usually equivalent, local supply and demand may vary them greatly.
Rolling for price variance might be done if there is no in-world reason why certain things might be more or
less expensive.
If that is the case the prices vary by \(((-1)^{1d10}\cdot1d10\cdot5) \%\).
Additionally everyone a character trades with has self interest and will keep some of
the tradevalue.\par
\begin{tabular}{cccccccc}
threshhold&0&5&8&11&14&17&20\\
kept tradevalue&50&33&20&10&5&2&0
\end{tabular}\\
Goods, at 100\% value each line is equivalent to one another and equivalent to 4 character creation points \par
\section{Tradegoods}\label{sec:tradegoods}
\begin{tabular}{|r|l|}
    \hline
    Storage & Article\\\hline
    10g & Elixir of Life (EOL)\\\hline
    100g & Alacast\\\hline
    500g & Experimental Tech Parts (ETP)\\\hline
    1kg & High Tech parts (HTP)\\\hline
    1.5kg & Advanced Gene Therapy Medicine (AGT)\\\hline
    5kg & Medium Tech parts (MTP)\\\hline
    10kg & Potent Medicine\\\hline
    50kg & Low Tech parts (LTP)\\\hline
    50kg & Seeking Rockets\\\hline
    100kg & Processed Medicine\\\hline
    100kg & Basic Rockets\\\hline
    100kg & Artillery\\\hline
    200kg & Basic Medicine\\\hline
    500kg & Improvised/Herbal Medicine\\\hline
    500kg & Base Tech parts (BTP)\\\hline
    500kg & Basic Ammunition\\\hline
    1000L & drinking Water (500days of drinking)\\\hline
    500L/35kg & Liquid Hydrogen\\\hline
    1t & Coal\\\hline
    350L & LiquidCombustionFuel\\\hline
    100kg & High Energy Rations (0.4kg/(day\(\cdot\)person))\\\hline
    1t & Normal Food (2.5 kg / (day\(\cdot\)person))\\\hline
    1 & Medium Quality Blueprint\\\hline
\end{tabular}\par
Open for more Suggestions\vspace{1.5cm}
Example:\par
Character A wants to get rid of 5 tonnes of Base Tech Scrap in favor of more easily transportable Alacast in a local
Dome.\par
If there are no storyelements influencing the prices, they are rolled.
First, the value of the BTP is determined to be (Roll: 6, 4) \(((-1)^{6}\cdot4\cdot5) \% = +20\%\).
Then the value of Alacast is determined
to be (Roll: 10, 3)\(((-1)^{10}\cdot3\cdot5) \% = +15\%\).
The result is that the rate between BTP and Alacast is
\( 500\cdot1.15/100\cdot1.2 = 4.79\) \\ Which means character As scrap is worth
\(5000 kg\cdot / 4.79 = 1043 g \) of Alacast.
He barters (with Resolve and Trade for 3, 2) with a local merchant and manages to
negotiate terms in which the merchant retains (Roll: 2, 4, 5, 8, 9 \(\Rightarrow 9 \Rightarrow 10\%\) of the Traded
value.
He decides to trade and gets 938 grams of Alacast.\vspace{1.5cm}
\pagebreak



\chapter{GameSystems}\label{ch:gamesystems}
\section{Health and Contamination}\label{sec:healthandcontamination}
\subsection{Healing}\label{subsec:healing}
For living entities damage is represented by wounds with the severity being the remainder of the damage after all
defenses were subtracted.
Humans roll a Fitness-Check once a wound starts healing.
For every met threshhold of the wound one level of regeneration rate for that particular wound is noted down.
If a wound is worsened/increased the Fitness-Check is repeated, selecting the worse of the two results.
If a wound is successfully and substantially improved, the Fitness Check is repeated, selecting the better of the two
results.
Regeneration can be aided or slowed by circumstances, such as therapy or environmental conditions.
Resonance of frequency 1 lowers the healing rate by its amplitude, even going negative, while resonance of frequency 10
improves healing rate by its amplitude.
Wounds do not necessarily lead to death or permanent impairments, but if they remain untreated for a long time, or there
is a runaway effect on one of them, the Storyteller may decree permanent repercussions, as appropriate, including Death.
Regeneration accumulates over successive days, and once the current severity is reached, the wound is lowered by 1
severity and regeneration is reset to 0.
If regeneration rate is negative, and the regeneration rate falls below 0, regeneration progress is set to the wound
severity and then the severity is increased by 1.
Standard threshholds are 2, 4, 6, 8, 10, 12, 14. \par
The following table lists healing time in days for severity and hit threshholds.\par
\begin{tabular}{c|rrrrrrrrrr}
    Severity & 1 & 2 & 3 & 4 & 5 & 6 & 7 & 8 & 9 & 10\\
    1 &  1 d & 1 d & 1 d & 1 d & 1 d & 1 d & 1 d & 1 d & 1 d & 1 d\\
    2 &  3 d & 2 d & 2 d & 2 d & 2 d & 2 d & 2 d & 2 d & 2 d & 2 d\\
    3 &  6 d & 4 d & 3 d & 3 d & 3 d & 3 d & 3 d & 3 d & 3 d & 3 d\\
    4 &  10 d & 6 d & 5 d & 4 d & 4 d & 4 d & 4 d & 4 d & 4 d & 4 d\\
    5 &  15 d & 9 d & 7 d & 6 d & 5 d & 5 d & 5 d & 5 d & 5 d & 5 d\\
    6 &  21 d & 12 d & 9 d & 8 d & 7 d & 6 d & 6 d & 6 d & 6 d & 6 d\\
    7 &  28 d & 16 d & 12 d & 10 d & 9 d & 8 d & 7 d & 7 d & 7 d & 7 d\\
    8 &  36 d & 20 d & 15 d & 12 d & 11 d & 10 d & 9 d & 8 d & 8 d & 8 d\\
    9 &  45 d & 25 d & 18 d & 15 d & 13 d & 12 d & 11 d & 10 d & 9 d & 9 d\\
    10 &  55 d & 30 d & 22 d & 18 d & 15 d & 14 d & 13 d & 12 d & 11 d & 10 d\\
    11 &  66 d & 36 d & 26 d & 21 d & 18 d & 16 d & 15 d & 14 d & 13 d & 12 d\\
    12 &  78 d & 42 d & 30 d & 24 d & 21 d & 18 d & 17 d & 16 d & 15 d & 14 d\\
    13 &  91 d & 49 d & 35 d & 28 d & 24 d & 21 d & 19 d & 18 d & 17 d & 16 d\\
    14 &  105 d & 56 d & 40 d & 32 d & 27 d & 24 d & 21 d & 20 d & 19 d & 18 d\\
    15 &  120 d & 64 d & 45 d & 36 d & 30 d & 27 d & 24 d & 22 d & 21 d & 20 d\\
    16 &  136 d & 72 d & 51 d & 40 d & 34 d & 30 d & 27 d & 24 d & 23 d & 22 d\\
    17 &  153 d & 81 d & 57 d & 45 d & 38 d & 33 d & 30 d & 27 d & 25 d & 24 d\\
    18 &  171 d & 90 d & 63 d & 50 d & 42 d & 36 d & 33 d & 30 d & 27 d & 26 d\\
    19 &  190 d & 100 d & 70 d & 55 d & 46 d & 40 d & 36 d & 33 d & 30 d & 28 d\\
    20 &  210 d & 110 d & 77 d & 60 d & 50 d & 44 d & 39 d & 36 d & 33 d & 30 d\\
    21 &  231 d & 121 d & 84 d & 66 d & 55 d & 48 d & 42 d & 39 d & 36 d & 33 d\\
    22 &  253 d & 132 d & 92 d & 72 d & 60 d & 52 d & 46 d & 42 d & 39 d & 36 d\\
    23 &  276 d & 144 d & 100 d & 78 d & 65 d & 56 d & 50 d & 45 d & 42 d & 39 d\\
    24 &  300 d & 156 d & 108 d & 84 d & 70 d & 60 d & 54 d & 48 d & 45 d & 42 d\\
    25 &  325 d & 169 d & 117 d & 91 d & 75 d & 65 d & 58 d & 52 d & 48 d & 45 d\\
\end{tabular}
\subsection{Treatment}\label{subsec:treatment}
Treating of wounds is usually done with a Competence or Theory + Healing Check, where medicine and equipment
can provide modifiers or advantages.
Many wounds require medicine to be treated at all.
Treating them without will either not be possible or generate disadvantage.
Standard threshholds are 8, 11, 14, 17, 20, with results below 5 having negative consequences.

Treating ingress wounds requires at least 10g of Alacast, with the treatment of
contamination inflicting the sum of all dice including and below or equal to a threshhold determined by the
Technology used (i.e.\ 8 for Saline-Alacast-Solution Injection) as damage.
The level of ingress wounds and contamination is directly lowered by the number of hit threshholds.

In all cases Alacast is used to precipitate Contamination from the body, the
crudest way is to introduce Alacast to the bloodstream to precipitate it inside the body and hope the body expels
it on its own, more refined ways are basically a dialysis or a very specific targetting of crystallization seeds
and retrieval of precipitation clusters.

\subsection{Medicine}\label{subsec:medicine}
Medicine is used while making a Treatment Check.
Each Treatment of a wound using Medicine consumes
severity\(\cdot\)100g of the medicine and provides the specific bonus.

\begin{enumerate}[label= -]
    \item \textbf{Naturopathy} uses healing and the respective field of the medicine.
    It provides a good floor and good threshhold bonuses.
    \item \textbf{Specialized Medicine}(name pending) live mixing of specialized cures from ingredients
    uses healing and the respective field of the ailment.
    It has no floor, but scales the best.
    \item \textbf{Broadband Medicine} is mass produced, refined medicine and uses a com\-bi\-nation of
    Healing, \-Scie\-nce(Med\-icine) and \-Red Biotech.
    It has a good \-floor, but low \-scaling.
    \item \textbf{Alacast} has no direct medicinal benefit, but effectively combats Contamination in all fields
    \item \textbf{Elixir of Life} is the top notch medicine, rejuvinates and heals nearly anything.
    Sadly, it contaminates the user.
\end{enumerate}

\subsection{Categories}\label{subsec:categories}
Every Character has a Contamination Resistance rating ranging from 1 to 5,
but in-world they are categorized (by supervivo) from A to C\@.
\begin{enumerate}[label = - ]
\item \textbf{CatA} Humans are usually Supervivo and usually have low Contamination Resistance
                    and high Affinity.
\item \textbf{CatB}s are outlaws, traders, hermits or for some other reason living outside the Domes and Tribes.
                    They can have any, but usually have medium Contamination Resistance and medium Affinity
\item \textbf{CatCs} are usually Tribals with high Contaminatin Resistance and low Affinity.
\end{enumerate}
\begin{tabular}{c|cc}
    Entity & Contamination Resistance  & Affinity requirements\\
    Experimental & 1 & 5 \\
    High Tech & 2 & 4\\
    Mid Tech & 3 & 3 \\
    Low Tech & 4 & 2\\
    Base Tech & 5 & 1\\
\end{tabular}


\subsection{Contamination}\label{subsec:contamination}
Contamination is a term used to describe the amalgamation of \-technology destroying \-nanoweapons, \-radiation, \-pollution
and general \-environmental \-hazards.
Contamination is air\-borne, waterborne, bloodborne and per\-meates pretty much everything in the environment.
It is usually assigned a \-level \-describing its intensity.
Only within the Domes of the Supervivo, \-inside a Mech or in a similarily \-purified environment can the \-
Contamination\-level ever reach 0. \par
Levels below 0 are necessary for taking apart Experimental Tech Technology, but getting an area so pure is hard.\par
\begin{enumerate}[label = - ]
\item Levels 1--3
are rare areas of low Contamination, like mines, airlocks, outdoor markets, outdoor settlements and so on.
Even subjected to this level of Contamination for months, it will not cause death.
However, for the more fragile Members of Society, the general quality of life can be greatly diminished.
\item Levels 4--6
are the most common and usually inhabited by Cat-B\@.
People without a form of permanent resistance can die here, but life expectancies should be a few months at least.
\item Levels 7--11
are usually inhabited by tribals but sometimes a few Cat-B have to make Camp here.
Anyone without contamination protection will die here within a few weeks.
\item Levels 12--15
are sparsely populated by hardy Tribals, but usually deserted.
\item Levels 16+
are the most hostile areas.
Weird Things happen here, but some say that there is great treasure\ldots
\end{enumerate}
There is no upper limit for contamination levels.
If an entity is in an area with contamination, every hour it receives the local contamination
as contamination damage.\par
\subsection{Contamination damage}\label{subsec:contamination-damage}
Anytime the an entity receives contamination damage, the character lowers that damage with a roll on their contamination
resistance and fitness plus all applicable modifiers (external only for contamination ingress).
Remaining Damage above 0 is applied to the ingress wound (extending the existing ingress wound if possible,
otherwise creating a new one) and raises contamination of that character by 1.
The contamination of the character is applied as an internal bonus on the contamination resistance result, but reduces
their healing rate, including going negative.

When a character directly interacts with an entity, the Charactercontamination*2
of that character is applied as Contamination damage, resisted by a Contaminationresistance of the Entity.
This is repeated every hour if the interaction continues.
While directly contacting an entity only the internal resistance applies, direct contact might be piloting a mech,
using a gun or shaking someones hand, except when proper seals are in place.
\subsection{Seals}\label{subsec:seals}
A seal of a level of X lowers the level of the Contamination behind it by X (they are usually from 6 upwards)
Contamination penetrates (and then slowly equalizes the Contamination) weak or damaged seals, broken seals offer a delay
at most.

Air has to be purified separately or enriched with oxygen.
(One human consumes about 500--600 litres of oxygen per day.) The Domes of the Supervivo for
example usually have a level 25 hermetical seal with purified and oxygenenriched air.


Examples of human sized external contamination modifiers or in-place options:


\begin{tabular}{cl}
    1& An alacast infused rag or a shut wooden door (although oxygen may run out)\\
    2& a breathermask or a simple wooden door airlock (again, oxygen)\\
    3& a partial gasmask or an airtight airlock\\
    4& a heavy partial gasmask with Goggles or a basetech air filtering facility\\
    5& a full gasmask\\
    6& as above + sealed clothing or lowtech air filtering facility\\
    7& as above + midtech scrubber (a little backpack)\\
    8& as above + hightech air scrubber or midtech air filtering facility\\
    9& as above + full mask with air from a compressed air tank\\
    10+& \hyperref[subsec:seals]{specially sealed} suit with airtank or hightech filtering facility\\
\end{tabular}






\chapter{Combat}\label{ch:combat}
\section{Rounds}\label{sec:combat-rounds}
Combat is done in rounds.
In Detailed Combat, each round is 5 seconds long, in Tactical Combat, rounds are of variable length and could
last an hour.
Every character present acts in turn (determined by circumstances or a Stealth, Perception, Willpower or other
situational check) and can usually move and act or attack.
Double actions, if possible, usually carry a 2 dice penalty on both actions, further actions carry even more penalties.

\section{Attacks}\label{sec:combat-attacks}
An attack has to be made on a valid target.
Targets are valid depending on what weapon is used, a direct fire weapon requires line of sight, for example.
Depending on the circumstances, the correct attack roll is made, usually involving dexterity, the weapon skill and
another skill depending on circumstance.
This roll, called attack value from here, is then modified by range depending on the weapon.

If the attack value is positive, the general vicinity of the target is hit.
Then, the enemy defense from dodging or cover is subtracted.

If the attack value is still positive, the target is hit and Damage is potentially inflicted,
Otherwise, the cover may still be damaged, depending on circumstance.


\section{Damage}\label{sec:combat-damage}
Entities have several hitzones, which zone specifically takes the is determined by the Entity and potential targeting.
Every successful hit will damage the hitzone, eventually leading to destruction.
A hits attack value is mitigated by armor.

If the remaining strength is above 0, the hitzone is damaged.

If it is under 0, the hitzone may still count as scratched.

Layered defenses do not provide protection if they are destroyed, some may already "leak" damage onto the sublayer
if they get damaged.

For a Biological Entity damage represents wounds and may heal.


\section{Death}\label{sec:combat-death}
There is no fixed numerical amount of life points, but every time a hitzone, including body parts, is damaged,
this damage is represented by some form of reduction in operation.
This can manifest in many ways, like penalty dice for associated activities, higher botch/malfunction frequencies,
lowered armor, and many more.
If there is a lot of damage, machines may become inoperable, people might pass out.
Death and Destruction only occur when the game master decrees so.

\section{Defense}\label{sec:combat-defense}
Defense can be gained from dodging or cover or other situationally appropriate means.
In general the defense value is just the roll of the action, if it would provide appropriate protection.
Taking cover behind a hanging fishing net or trying to talk down an automated turret might not result in any defense
at all.

\textbf{Dodging}, usually done with agility and acrobatics, footwork or running, requires the action for that round,
meaning it needs to be repeated every round, and is subject to double action rules.

\textbf{Cover}, once taken with agility and tactics, running, stealth, footwork or whatever is appropriate,
provides its defense until it is nullified by flanking or movement is taken.
Furthermore, there is the possibility of full cover granting a flat bonus onto the roll

\textbf{Other Actions} may include throwing sand, building rapport, talking someone out of shooting and are resolved
by themselves


\section{Detailed Combat}\label{sec:detailed-combat}
\begin{enumerate}
    \item Every participant (technically in secret) decides on what they are going to do.
    \item Non-offensive actions are resolved first, then offensive actions in descending order of rolled result.
\end{enumerate}



\section{Tactical Combat}\label{sec:tactical-combat}
In many respects much like detailed combat, Tactical Combat zooms out over encounters that could take a while,
it is not unusual to switch back and forth.
In Tactical Combat, there are usually no single actions, Tactical Actions encompass states and processes.
Anytime something changes, everyone who is aware may change what they are doing.
If these reactions cascade, detailed combat is the natural consequence and actions will proceed in 5 second intervals.


If an actor does more or less the same thing 3 times in a row, it might be appropriate to enter Tactical Combat.
A state and/or process will interpret the last 3 rolls and take the average of that number.
Rolls that do not succeed fully may carry over some part of the result until the threshold is met, so someone nearly
missing will instead hit at a lower rate


When entering Tactical Combat some form of time commitment is agreed upon and after each interval one roll is made.
The moment of this roll is also the only time for unprovoked re-evaluation and change of plans.
The interval should be changed by the Storyteller depending on the density of action.
Tactical Actions might include \par
\begin{enumerate}
    \item firing on an enemy or a position
    \item guarding a position
    \item breaking down a door
    \item traversing to a point
    \item looking out
\end{enumerate}

\section{Targeting}\label{sec:combat-targeting}
Generally a center-mass targeting is assumed.
From there deviations occur, spreading the damage over neighboring parts
For a Human target, the hitzones are determined by d10, or by the resonance of the attack roll if any.
If there are several resonances, the attacker gets to pick.


\begin{tabular}{cc}
    roll & target\\\hline
    1,2,3 & torso \\
    4,5,6 & legs \\
    7,8 & arms \\
    9 & hands or feet \\
    10 & head
\end{tabular}

\subsection{Called Shots}\label{subsec:combat-calledshots}
To ensure that a specific area is hit, a called shot can be made, risking missing altogether.
A Penalty to the attack value is taken, to hit the specific body part.
A resonance with a number corresponding to a body part less than what was aimed for,
will instead attack that part, as above.

\begin{tabular}{cc}
    target & penalty\\\hline
    head & 3 \\
    hands or feet & 2 \\
    legs & 1 \\
    torso& 1
\end{tabular}

\section{Weapons}\label{sec:weapons}

Weapons have the following stats:
\begin{enumerate}
    \item \textbf{Range} - Minimum / Medium / Drop, in meters
        \begin{enumerate}
            \item \textbf{Minimum} - below which the weapon is hard to use in combat and penalty dice are issued
            \item \textbf{Optimal} - below which no penalties for range are in effect
            \item \textbf{Drop} - every how many meters a -1 penalty on shots is incurred
        \end{enumerate}
    \item \textbf{Ammunition} - how many shots per reload/how much weight per reload
    \item \textbf{Shot Interval} - in seconds,  values below 5 allow multiple shots per round
    \item \textbf{Handling} - initiative bonus
    \item \textbf{Mods} - as per mod
    \item[Crit] - requisites and effects of a critical hit
    \item \textbf{Skill} - the main skill to use this weapon
\end{enumerate}

\subsection{Example weapons used in Play}\label{subsec:example-weapons-used-in-play}
To be categorized and refined later.
These Weapons are of Personal Size.
\subsubsection{Sniper Rifle}
200[7,9,11]10\par
500[5,7,9,11,13,15,17,19]5\par
1500[7,9,11,14,17,20]2\par
6000[15,20]0.5\par
Costs: 1.5TU\par
Damage: 20\par
\hyperref[subsec:hit]{\textbf{Hit Calculation}} \par
[0,80,100,150]
Ammo 100g/shot

\subsubsection{Heavy Revolver}
3[5,7,9,11,13]2\par
20[5,7,9,11,13,15,17]10\par
50[9,14,18]5\par
100[12,16]5\par
150[18]3\par
Costs: 0.1TU\par
Damage: 15\par
Ammo 50g/shot

\subsubsection{Light Revolver}
3[5,7,9,11,13]2\par
20[5,7,9,11,13]10\par
50[7,12,16]5\par
100[14,17]5\par
Costs: 0.1TU\par
Damage: 10\par
Ammo 20g/shot

\subsubsection{Shotgun Revolver}
Slugs:\par
3[3,5,7,9,11,13]10\par
15[5,7,9,11,13]10\par
25[7,12,16]5\par
50[14,17]5\par
Shot:\par
3[3,5,7,9,11,13]10\par
36[5,8,11,14,17]
\hyperref[subsec:hit]{\textbf{Hit Calculation}} \par
[10,20,30,40,50,60,70,80,90,100]
Costs: 0.4TU\par
Damage: 20\par
Ammo 20g/shot


\subsubsection{Wooden Bow}
20[5,7,9,11,13,15,17,19,20]3\par
50[7,10,13,16,19]2\par
100[10,15,20]1\par
200[15,18,20]0.1\par
Costs: 0.1TU\par
Damage: 10\par
\hyperref[subsec:hit]{\textbf{Hit Calculation}} \par
[0,80,100,150]
Ammo 100g/shot

\subsubsection{Tooth Spear}
3[5,7,9,11,13,15,17,19,20]\par
Costs: 0.2TU\par
Damage: 30\par

\subsubsection{Obsidian Knife}
1[5,9,13,14,15,16,17,18,19,20]\par
Costs: 0.2TU\par
Damage: 20 \par
\hyperref[subsec:hit]{\textbf{Hit Calculation}} \par
[0,1,10,20,80,100,150]






\chapter{Tech}\label{ch:tech}
Electrolysis splits 1 L H2O into 622.22 L Oxygen und 1244.4 using 3.7037 kwh\\
\section{Engineering}\label{sec:engineering}
\subsection{Blueprints}\label{subsec:blueprints}
Blueprints are required for Engineer(Build) tasks.
To create a blueprint a character needs the appropriate materials, usually a computer of some sort, but even sand and a
stick can work.
The time required is based on the complexity of the project.
To find out if the blueprint is within the capabilities of the character check their specific knowledge + Theory
against the difficulty of the project.
If at least one threshhold is hit, the blueprint is viable.
Starting quality of a blueprint can be negative.
Every threshhold hit increases the quality of the blueprint by 1.
If the character does not have the Appropriate Knowledge he can check sufficiently similar knowledge with
an appropriate penalty dice (1 per techlevel, 3 for different module classes on the same techlevel)
Even a blueprint with a negative bonus is useful, since it enables building the item in question at all.\\
A blueprint states the name of the Item, its category and the required materials.

\subsection{Building}\label{subsec:building}
Building something requires a blueprint, the raw materials and the necessary tools.
Check Engineering modified by the blueprint and the conditions.
The finished product is [check result \(\cdot\) 10\%] Efficient.
