//TODO: General format and make words good yes.
Every sublayer has up to 3 levels, only one sublayer can ever be on level 3, the rest are capped at level 2,
transfer can be done.
Every superlayer has up to 3 levels, and can only be increasedby 1 for every related sublayer that is at level 2 or 3
\subsection{Mecha Weapons Expertise}\label{subsec:mecha-weapons-expertise}
Mecha Weapons means any weapon too big to be carried, including stationary turrets.
Usually Weapons are fired with \hyperref[itm:focus]{Focus}.
\subsubsection{Laser}
Laser or technically "Directed Energy Weapons" \par
are a group of weapons which typically do not use Ammunition, but raw Energy to inflict their damage.
Their comparably low damage is made up for by being as accurate as a light beam,
plus, the high amounts of particles in the air make for pretty lights.
\subsubsection{Projectile}
Projectile Weapons or sometimes just "Guns" \par
are a group of weapons which discharge Projectiles towards a target.
Ammunition has to be employed, but there are several different types available.
\subsubsection{Missile}
Missile Weapons are a weapons that travel to the target to inflict damage, the difference
to Projectile Weapons is that Missile Weapons are usually self propelled instead of being accelerated at their
startpoint.
Missiles are comparatively heavy and expensive, but most of them ignore shields and/or deliver
special devices or bombs to the target.
\subsubsection{Attack Modes}
\begin{enumerate}[label= -]
    \item \textbf{Direct} Weapons are many and varied but share the characteristic of being pointed roughly at
    the target they are shooting.
    They require line of sight.
    \item \textbf{Indirect} Weapons are just as varied as Direct Weapons but have the key difference
    of not pointing at the target.
    Usually Up and in an angle, but not always.
    They require information about the targets position, but are free from line of sight restrictions, as long as
    the Attackstill has a valid way to travel to the target.
    \item \textbf{Seeking} Weapons have some sort of sensors or a dataconnection, and will follow a Target, as long as the Lock
    remains unbroken.
    This skill mainly deals with the correct usage and parametrization of such Weapons.
    \item \textbf{Beam} Weapons that emit long, sometimes even sustained streams, making them easier to aim
    and inflicting more damage, usually at the price of higher, sustained costs and less flexibility.
\end{enumerate}
\subsubsection{Melee}
Melee Weapons of mech sizes are capable of inflicting utter destruction, but are usually
limited to close ranges, which makes them not very useful at long distances.
\begin{enumerate}[label= -]
    \item \textbf{Sharp} Melee Weapons have slightly longer range and are faster and usually do good damage.
    Most of them use the Movement System to get ready to attack again, so they may attack a lot of times on fast mechs.
    They also are the most precise of the melee weapons.
    \item \textbf{Blunt} Melee Weapons are usually slow, and easier to dodge, but their raw damage output is only
    matched by Artillery.
    \item \textbf{Unarmed} is not technically a category of weapon, but it represents being able to fight in melee
    without any special weapon, which allows for mostly disabling maneuvers.
\end{enumerate}
\subsection{Piloting Practice}\label{subsec:piloting-practice}
\begin{enumerate}[label= -]
    \item \textbf{Plain} Terrain
    \item \textbf{Desert} Terrain
    \item \textbf{Forest} Terrain
    \item \textbf{Arctic} Terrain
    \item \textbf{Mountain} Terrain
    \item \textbf{Urban} Terrain
    \item \textbf{Mud} Terrain
    \item \textbf{Marine} Terrain
    \item \textbf{Extreme} Terrain
    \item \textbf{Exo} is short for "Exoskeleton" and refers to Mechs that are barely larger than an extended Bodyarmor
    or Mechs that are generally not much bigger than their Pilot. (<1t)
    \item \textbf{Tiny} a versatile spot for Contraptions between Exo and Very Light, these include buggies, cars and
    a lot of small, fast Vehicles. (1-5t)
    \item \textbf{Very Light} include things like APCs and small fast walkers and light tanks.(5-15t)
    \item \textbf{Light} is about as big as a medium to main tank. (15-50 t)
    \item \textbf{Medium} is about as big as a heavy to superheavy tanks (50-150 t)
    \item \textbf{Heavy} is above superheavy tanks. (150-500t)
    \item \textbf{Very Heavy} LANDKREUZER (500t-1500t)
    \item \textbf{Ultra} ???
\end{enumerate}
\subsection{Aux}\label{subsec:aux}
\subsubsection{Defense}
\begin{enumerate}[label= -]
    \item \textbf{Decoys}
    \item \textbf{Shields}
    \item \textbf{Deflectors}
\end{enumerate}
\subsubsection{Offense}
\begin{enumerate}[label= -]
    \item \textbf{Explosives}
    \item \textbf{Sniping}
    \item \textbf{Gadgets}
\end{enumerate}
\subsubsection{Recon}
\begin{enumerate}[label= -]
    \item \textbf{Scanners}
    \item \textbf{Suppressors}
    \item \textbf{Sabotage}
\end{enumerate}
\subsubsection{Utility}
\begin{enumerate}[label= -]
    \item \textbf{Communications}
    \item \textbf{Reprocessing}
    \item \textbf{Movement}
\end{enumerate}
\par