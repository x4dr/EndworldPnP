\section{Life}\label{sec:life}
If an entity gets damaged, the damage is decreased by the result of the roll made by the outermost layer of defense.
If the remaining damage is above 0, that remaining damage is usually passed on to the layer below.
Layers can fail, and then no longer participate during damage prevention.
If damage passes all layers of defense,
the remaining numerical value of the damage is noted down as the severity of the damage.
For a Biological Entity damage is wounds an may heal..
\section{Death}\label{sec:death}
There is no fixed numerical amount of lifepoints, but every wound/instance of damage will impede the Entity.
This can manifest in many ways, like penalty dice for associated activities, higher botch/malfunction frequencies,
and many more.
If there is a lot of damage, machines may become inoperable, people might pass out.
Death and Destruction only occur when the Gamemaster decrees so.
\section{Rounds}\label{sec:rounds}
Combat is done in rounds.
In Detailed Combat, each round is 5 seconds long, in Tactical Combat, rounds are of variable length and could
last an hour.
Every character present acts in turn (determined by circumstances or a Stealth, Perception, Willpower or other
situational check) and can usually move and do an action (an attack or action that can be done in the remaining time).
Any action beyond the first carries with it a penalty of die and might be flat impossible depending on circumstance.
Weapons that do not explicitly state how many times a turn they can be fired, can only be fired once per turn, switching
weapons is an action that - without specific perks - uses the whole hand(s) for the remainder of the turn, and so on.

\section{Detailed Combat}\label{sec:detailed-combat}
\begin{enumerate}
    \item Every participant (technically in secret) decides on what they are going to do.
    \item Each defense is determined, depending on the current action of the participant.
    \item Each offense is determined, depending on the currect action and target of the participant.
    \item Non-offensive Actions are resolved first, then offensive actions in descending Order of rolled result.
\end{enumerate}

\section{Tactical Combat}\label{sec:tactical-combat}
In many respects much like detailed combat, Tactical Combat zooms out over encounters that could take a while,
it is not unusual to switch back and forth.
In Tactical Combat, there are usually no single actions, Tactical Actions (which encompass states and processes).
Anytime something changes, everyone with intel on that change gets to reevaluate and change what they are doing.
If these reactions cascade, detailed combat is the natural consequence and actions will procede in 5 second intervals.
If an actor does more or less the same thing 3 times in a row, it might be appropriate to enter Tactical Combat.\par
A state and/or process will interprete the last 3 rolls and take the average of that number.
When entering Tactical Combat some form of time commitment is agreed upon and after each interval one roll is made.
The moment of this roll is also the only time for unprovoked re-evaluation and change of plans.
The interval should be changed by the Storyteller depending on the density of action.
Tactical Actions might include \par
\begin{enumerate}
    \item firing on an enemy or a position
    \item guarding a position
    \item breaking down a door
    \item traversing to a point
    \item looking out
\end{enumerate}

\subsection{Defense}\label{subsec:defense}
Defense is usually (in the case of dodging) checked with Agility and
\begin{enumerate}[label = - ]
    \item \hyperref[subsec:tactics]{\textbf{tactics}} on ordered advance/retreat
    \item \hyperref[subsec:instinct]{\textbf{instinct}} in general mayhem
    \item \hyperref[subsec:footwork]{\textbf{footwork}} in close range
    \item \hyperref[subsec:running]{\textbf{running}} when zig zagging
    \item anything else pertaining to the situation
\end{enumerate}
Evasion levels are determined with the result of the defensive interpretation of the combat roll
and the character specific evasive threshhold set. (standard unarmored is [5,7,9,11,13])

\subsection{Offense}\label{subsec:offense}
Offense level are determined with the result of the offensive \-interpretation of the combat roll \-and the range and
weapon \-specific threshhold set.\par
\subsubsection{Aiming}\label{subsubsec:aiming}
Aiming for a specific bodypart gives control over the location of the inflicted wound, but it also carries a penalty
that is added to the evasion levels of the target

\begin{tabular}{cc}
    target & penalty\\\hline
    head & 3 \\
    hand & 2 \\
    legs & 1 \\
    center mass & 0
\end{tabular}

\subsection{Hit Calculation}\label{subsec:hit}
If the offense level of an attack is at least 1, the hit roll is made.
It consists of a singular die, with the offense levels as bonus dice, and the defense levels as penalty dice,
see \hyperref[sec:bonus--and-penaltydice]{Bonus- and Penaltydice}.
The hit quality will be a number from 1 to 10. \par
For most direct projectile weapons the below table is a guideline.
The encoded form would be [0,1,10,20,50,100,100]

\begin{tabular}{c|cr}
    hit quality & result & damage modifier \\\hline
    1 & far miss & 0\%\\
    2 & miss& 0\%\\
    3 & close miss& 0\%\\
    4 & near miss& 0\%\\
    5 & tangential& 1\%\\
    6 & graze& 10\%\\
    7 & slight hit& 20\%\\
    8 & partial hit& 50\%\\
    9 & hit& 100\%\\
    10 & full hit& 100\%\\
\end{tabular}

On a full hit, the attacker inflicts the wound were planned
The specific damage to apply will be listed with the weapon.

\section{Weapons}\label{sec:weapons}

Weapons have the following stats:
\begin{enumerate}
    \item \textbf{Range} - Minimum / Medium / Drop, in meters
        \begin{enumerate}
            \item \textbf{Minimum} - below which the weapon is hard to use in combat and penalty dice are issued
            \item \textbf{Optimal} - below which no penalties for range are in effect
            \item \textbf{Drop} - every how many meters a -1 penalty on shots is incurred
        \end{enumerate}
    \item \textbf{Ammunition} - how many shots per reload/how much weight per reload
    \item \textbf{Shot Interval} - in seconds,  values below 5 allow multiple shots per round
    \item \textbf{Handling} - initiative bonus
    \item \textbf{Mods} - as per mod
    \item[Crit] - requisites and effects of a critical hit
    \item \textbf{Skill} - the main skill to use this weapon
\end{enumerate}

\subsection{Example weapons used in Play}\label{subsec:example-weapons-used-in-play}
To be categorized and refined later.
These Weapons are of Personal Size.
\subsubsection{Sniper Rifle}
200[7,9,11]10\par
500[5,7,9,11,13,15,17,19]5\par
1500[7,9,11,14,17,20]2\par
6000[15,20]0.5\par
Costs: 1.5TU\par
Damage: 20\par
\hyperref[subsec:hit]{\textbf{Hit Calculation}} \par
[0,80,100,150]
Ammo 100g/shot

\subsubsection{Heavy Revolver}
3[5,7,9,11,13]2\par
20[5,7,9,11,13,15,17]10\par
50[9,14,18]5\par
100[12,16]5\par
150[18]3\par
Costs: 0.1TU\par
Damage: 15\par
Ammo 50g/shot

\subsubsection{Light Revolver}
3[5,7,9,11,13]2\par
20[5,7,9,11,13]10\par
50[7,12,16]5\par
100[14,17]5\par
Costs: 0.1TU\par
Damage: 10\par
Ammo 20g/shot

\subsubsection{Shotgun Revolver}
Slugs:\par
3[3,5,7,9,11,13]10\par
15[5,7,9,11,13]10\par
25[7,12,16]5\par
50[14,17]5\par
Shot:\par
3[3,5,7,9,11,13]10\par
36[5,8,11,14,17]
\hyperref[subsec:hit]{\textbf{Hit Calculation}} \par
[10,20,30,40,50,60,70,80,90,100]
Costs: 0.4TU\par
Damage: 20\par
Ammo 20g/shot


\subsubsection{Wooden Bow}
20[5,7,9,11,13,15,17,19,20]3\par
50[7,10,13,16,19]2\par
100[10,15,20]1\par
200[15,18,20]0.1\par
Costs: 0.1TU\par
Damage: 10\par
\hyperref[subsec:hit]{\textbf{Hit Calculation}} \par
[0,80,100,150]
Ammo 100g/shot

\subsubsection{Tooth Spear}
3[5,7,9,11,13,15,17,19,20]\par
Costs: 0.2TU\par
Damage: 30\par

\subsubsection{Obsidian Knife}
1[5,9,13,14,15,16,17,18,19,20]\par
Costs: 0.2TU\par
Damage: 20 \par
\hyperref[subsec:hit]{\textbf{Hit Calculation}} \par
[0,1,10,20,80,100,150]




