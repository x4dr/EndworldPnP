\section{Rounds}\label{sec:combat-rounds}
Combat is done in rounds.
In Detailed Combat, each round is 5 seconds long, in Tactical Combat, rounds are of variable length and could
last an hour.
Every character present acts in turn (determined by circumstances or a Stealth, Perception, Willpower or other
situational check) and can usually move and act or attack.
Double actions, if possible, usually carry a 2 dice penalty on both actions, further actions carry even more penalties.

\section{Attacks}\label{sec:combat-attacks}
An attack has to be made on a valid target.
Targets are valid depending on what weapon is used, a direct fire weapon requires line of sight, for example.
Depending on the circumstances, the correct attack roll is made, usually involving dexterity, the weapon skill and
another skill depending on circumstance.
This roll, called attack value from here, is then modified by range depending on the weapon.

If the attack value is positive, the general vicinity of the target is hit.
Then, the enemy defense from dodging or cover is subtracted.

If the attack value is still positive, the target is hit and Damage is potentially inflicted,
Otherwise, the cover may still be damaged, depending on circumstance.


\section{Damage}\label{sec:combat-damage}
Entities have several hitzones, which zone specifically takes the is determined by the Entity and potential targeting.
Every successful hit will damage the hitzone, eventually leading to destruction.
A hits attack value is mitigated by armor.

If the remaining strength is above 0, the hitzone is damaged.

If it is under 0, the hitzone may still count as scratched.

Layered defenses do not provide protection if they are destroyed, some may already "leak" damage onto the sublayer
if they get damaged.

For a Biological Entity damage represents wounds and may heal.


\section{Death}\label{sec:combat-death}
There is no fixed numerical amount of life points, but every time a hitzone, including body parts, is damaged,
this damage is represented by some form of reduction in operation.
This can manifest in many ways, like penalty dice for associated activities, higher botch/malfunction frequencies,
lowered armor, and many more.
If there is a lot of damage, machines may become inoperable, people might pass out.
Death and Destruction only occur when the game master decrees so.

\section{Defense}\label{sec:combat-defense}
Defense can be gained from dodging or cover or other situationally appropriate means.
In general the defense value is just the roll of the action, if it would provide appropriate protection.
Taking cover behind a hanging fishing net or trying to talk down an automated turret might not result in any defense
at all.

\textbf{Dodging}, usually done with agility and acrobatics, footwork or running, requires the action for that round,
meaning it needs to be repeated every round, and is subject to double action rules.

\textbf{Cover}, once taken with agility and tactics, running, stealth, footwork or whatever is appropriate,
provides its defense until it is nullified by flanking or movement is taken.
Furthermore, there is the possibility of full cover granting a flat bonus onto the roll

\textbf{Other Actions} may include throwing sand, building rapport, talking someone out of shooting and are resolved
by themselves


\section{Detailed Combat}\label{sec:detailed-combat}
\begin{enumerate}
    \item Every participant (technically in secret) decides on what they are going to do.
    \item Non-offensive actions are resolved first, then offensive actions in descending order of rolled result.
\end{enumerate}



\section{Tactical Combat}\label{sec:tactical-combat}
In many respects much like detailed combat, Tactical Combat zooms out over encounters that could take a while,
it is not unusual to switch back and forth.
In Tactical Combat, there are usually no single actions, Tactical Actions encompass states and processes.
Anytime something changes, everyone who is aware may change what they are doing.
If these reactions cascade, detailed combat is the natural consequence and actions will proceed in 5 second intervals.


If an actor does more or less the same thing 3 times in a row, it might be appropriate to enter Tactical Combat.
A state and/or process will interpret the last 3 rolls and take the average of that number.
Rolls that do not succeed fully may carry over some part of the result until the threshold is met, so someone nearly
missing will instead hit at a lower rate


When entering Tactical Combat some form of time commitment is agreed upon and after each interval one roll is made.
The moment of this roll is also the only time for unprovoked re-evaluation and change of plans.
The interval should be changed by the Storyteller depending on the density of action.
Tactical Actions might include \par
\begin{enumerate}
    \item firing on an enemy or a position
    \item guarding a position
    \item breaking down a door
    \item traversing to a point
    \item looking out
\end{enumerate}

\section{Targeting}\label{sec:combat-targeting}
Generally a center-mass targeting is assumed.
From there deviations occur, spreading the damage over neighboring parts
For a Human target, the hitzones are determined by d10, or by the resonance of the attack roll if any.
If there are several resonances, the attacker gets to pick.


\begin{tabular}{cc}
    roll & target\\\hline
    1,2,3 & torso \\
    4,5,6 & legs \\
    7,8 & arms \\
    9 & hands or feet \\
    10 & head
\end{tabular}

\subsection{Called Shots}\label{subsec:combat-calledshots}
To ensure that a specific area is hit, a called shot can be made, risking missing altogether.
A Penalty to the attack value is taken, to hit the specific body part.
A resonance with a number corresponding to a body part less than what was aimed for,
will instead attack that part, as above.

\begin{tabular}{cc}
    target & penalty\\\hline
    head & 3 \\
    hands or feet & 2 \\
    legs & 1 \\
    torso& 1
\end{tabular}

\section{Weapons}\label{sec:weapons}

Weapons have the following stats:
\begin{enumerate}
    \item \textbf{Range} - Minimum / Medium / Drop, in meters
        \begin{enumerate}
            \item \textbf{Minimum} - below which the weapon is hard to use in combat and penalty dice are issued
            \item \textbf{Optimal} - below which no penalties for range are in effect
            \item \textbf{Drop} - every how many meters a -1 penalty on shots is incurred
        \end{enumerate}
    \item \textbf{Ammunition} - how many shots per reload/how much weight per reload
    \item \textbf{Shot Interval} - in seconds,  values below 5 allow multiple shots per round
    \item \textbf{Handling} - initiative bonus
    \item \textbf{Mods} - as per mod
    \item[Crit] - requisites and effects of a critical hit
    \item \textbf{Skill} - the main skill to use this weapon
\end{enumerate}

\subsection{Example weapons used in Play}\label{subsec:example-weapons-used-in-play}
To be categorized and refined later.
These Weapons are of Personal Size.
\subsubsection{Sniper Rifle}
200[7,9,11]10\par
500[5,7,9,11,13,15,17,19]5\par
1500[7,9,11,14,17,20]2\par
6000[15,20]0.5\par
Costs: 1.5TU\par
Damage: 20\par
\hyperref[subsec:hit]{\textbf{Hit Calculation}} \par
[0,80,100,150]
Ammo 100g/shot

\subsubsection{Heavy Revolver}
3[5,7,9,11,13]2\par
20[5,7,9,11,13,15,17]10\par
50[9,14,18]5\par
100[12,16]5\par
150[18]3\par
Costs: 0.1TU\par
Damage: 15\par
Ammo 50g/shot

\subsubsection{Light Revolver}
3[5,7,9,11,13]2\par
20[5,7,9,11,13]10\par
50[7,12,16]5\par
100[14,17]5\par
Costs: 0.1TU\par
Damage: 10\par
Ammo 20g/shot

\subsubsection{Shotgun Revolver}
Slugs:\par
3[3,5,7,9,11,13]10\par
15[5,7,9,11,13]10\par
25[7,12,16]5\par
50[14,17]5\par
Shot:\par
3[3,5,7,9,11,13]10\par
36[5,8,11,14,17]
\hyperref[subsec:hit]{\textbf{Hit Calculation}} \par
[10,20,30,40,50,60,70,80,90,100]
Costs: 0.4TU\par
Damage: 20\par
Ammo 20g/shot


\subsubsection{Wooden Bow}
20[5,7,9,11,13,15,17,19,20]3\par
50[7,10,13,16,19]2\par
100[10,15,20]1\par
200[15,18,20]0.1\par
Costs: 0.1TU\par
Damage: 10\par
\hyperref[subsec:hit]{\textbf{Hit Calculation}} \par
[0,80,100,150]
Ammo 100g/shot

\subsubsection{Tooth Spear}
3[5,7,9,11,13,15,17,19,20]\par
Costs: 0.2TU\par
Damage: 30\par

\subsubsection{Obsidian Knife}
1[5,9,13,14,15,16,17,18,19,20]\par
Costs: 0.2TU\par
Damage: 20 \par
\hyperref[subsec:hit]{\textbf{Hit Calculation}} \par
[0,1,10,20,80,100,150]




